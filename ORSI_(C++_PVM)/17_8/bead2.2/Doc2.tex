\documentclass[12pt]{article}

%margó méretek zakdogában 32mm
\usepackage[a4paper,
inner = 25mm,
outer = 25mm,
top = 25mm,
bottom = 25mm]{geometry}

\usepackage{lmodern}
\usepackage[magyar]{babel}
\usepackage[utf8]{inputenc}
\usepackage[T1]{fontenc}
\usepackage[hidelinks]{hyperref}
\usepackage{graphicx}
\usepackage{amssymb}
\usepackage{epstopdf}
\usepackage{setspace}
\usepackage[nottoc,numbib]{tocbibind}
\usepackage{setspace}
\usepackage[bottom]{footmisc}
\usepackage{adjustbox}

\setstretch{1.2}
\begin{document}
	
\begin{titlepage}
	\vspace*{0cm}
	\centering
	\begin{tabular}{cp{1cm}c}
		\begin{minipage}{4cm}
			\vspace{0pt}
			\includegraphics[width=1\textwidth]{elte_cimer}
		\end{minipage} & &
		\begin{minipage}{7cm}
			\vspace{0pt}Eötvös Loránd Tudományegyetem \vspace{10pt} \newline
			Informatikai Kar \vspace{10pt} \newline
			Programozási Nyelvek és Fordítóprogramok Tanszék
		\end{minipage}
	\end{tabular}
	
	\vspace*{0.2cm}
	\rule{\textwidth}{1pt}
	
	\vspace*{3cm}
	{\Huge Osztott rendszerek specifikációja és implementációja }
	
	\vspace*{0.5cm}
	{\normalsize IP-08bORSIG}
	
	\vspace{2cm}
	{\huge Dokumentáció az 2. beadandóhoz}
	
	\vspace*{5cm}
	
	{\large \verb|Mikus Márk István| } % (név)
	
	{\large \verb|CM6TSV| }  % (neptun)
		
	
	\vfill
	
	\vspace*{1cm}
	2017. november 21. % (dátum)
\end{titlepage}

\section{Kitűzött feladat}
Feladatunk annak eldöntése, hogy egy adott halmaznak létezik e olyan részhalmaza, melyben található elemek összege pontosan megegyezik egy előre megadott számmal!

A szükséges adatokat a program 3 parancssori paraméteren keresztül kapja még. Az első, egy egész értékű adat, mely a feladatban definiált számot jelzi, ezt kell valamilyen módon elérni a halmazelemek összegével. A második, egy fájl neve - ez tartalmazza a kiinduló halmaz elemeit (inputfájl). A felépítése az alábbi: 

A fájl első sorában egy nemnegatív egész szám (N) áll - a kiinduló halmazunk elemszáma tehát N. A következő sorban összesen N db egész számot olvashatunk (pozitív és negatív egyaránt), melyek a halmaz elemeit jelölik (sorrendiséget nem kötünk meg köztük).

Egy megfelelő bemeneti fájl (például data.txt) ekkor: \\
\verb|6| \\
\verb|3 34 4 17 5 2 |\\
A harmadik paraméter, annak a fájlnak a neve, melybe a feladat megoldása során kapott választ kell írni.

A kimeneti fájlban található információ az alábbi legyen:\footnote{Egyik esetben sem kell a sor végére enter!}
\begin{itemize}
    \item létezik N összegű részhalmaz: \verb|May the subset be with You!|
    \item nem létezik ilyen részhalmaz: \verb|I find your lack of subset disturbing!|
\end{itemize}
A program egy lehetséges futtatása így az alábbi módon valósulhat meg:

\verb|pvm> spawn -> master 12 data.txt result.txt| \\
A kapott kimenet (result.txt) az előző bemenetre tehát a következő:

\verb|May the subset be with you!|\footnote{A 3, 4 és 5 elemek összege pontosan 12, így tudunk ilyen halmazt mutatni.} \\

A feladatot \textit{Divide \& Conquer} módszer alapján PVM3 használatával kell megoldani, az alábbiak alapján:
Amennyiben az elvárt összeg 0, alapesetnél vagyunk, ezt ki tudjuk elégíteni az üres halmaz, mint részhalmaz segítségével, tehát $igaz$.
Amennyiben az elvárt összeg nem nulla, de a halmazunk üres, erre nem tudunk megoldást adni, az eredmény $hamis$.
Egyéb esetben két lehetőségünk van, az elérni kívánt (összegű) halmazba a jelenleginek egy tetszőleges elemét vagy beletesszük, vagy nem.
Ha igen, akkor rekurzívan nézzük meg, hogy az a halmaz kielégíthető -e, melybe a most vizsgált elemet nem vesszük bele. Tehát pl. egy 10 összegű, $K$ elemszámú halmaz akkor és csak akkor tartalmazza a 3-at, mint elemet, ha a 7 összegű, $K-1$ elemszámú halmazt ki tudjuk elégíteni (amely az itt vizsgált 3-at, mint elemet nem tartalmazza) a feltételek alapján.
Ha nem, akkor a rekurzió azt dönti el, hogy egy kisebb elemszámú, de azonos súlyú halmazra teljesül e az elvárt feltétel. Ha az 5 összegű, $K$ elemű halmazból ki akarjuk venni a 35-öt, mint elemet, de a vizsgált súlyt nem akarjuk változatatni, akkor vizsgáljuk meg az 5 összegű, $K-1$ elemű halmazt (a 35 nélkül).
Ha a rekurzió valamelyik ága igazat ad, akkor a válaszunk $igaz$, tehát tudunk ilyen halmazt mondani (az elemeket nem kell megadni), egyéb esetben $hamis$.

\section{Felhasználói dokumentáció}

A program egy konzolból futtatható alkalmazás, amelyet a feladat által leírt módon kell paraméterezni, azaz az első paraméter egy egész szám, a második az input fájl neve és a harmadik a kimenő állomány neve amibe a program bele írja az input adatokra adott válaszát.

\subsection{Rendszer-követelmények, telepítés}

A programunk több platformon is futtatható ($Windows/Linux$), amelyen fel van telepítve a PVM3 rendszer, amelynek része egy függvénykönyvtár, a démon és a konzol.

\subsection{A program használata}

A programot  a PVM konzolon keresztül lehet futtatni. A PVM indításához szükséges parancs: 

\verb|pvm [Optional:] <hostsfile name>|\footnote{Amennyiben megadjuk az opcionális paramétert, úgy a PVM-konzol a megadott fájlban meghatározott configurációval indul, s csatlakoztatja a fájlban felsorolt $blade$-ket} \\
Ennek hatására elindul a PVM-konzol. Ezek után a $spawn$ kulcsszóval lehet meghívni a szűlőfolyamatot az alábbi módon: 

\verb|pvm> spawn -> master <szám> <input> <output>|
Ahol a \verb|<szám>|, a feladatban szerepelő keresett össteg, az \verb|<input>| a bemeneti fájl, míg az \verb|<output>| a kimeneti fájl neve.

A program hívásakor, a $~/pvm3/$ könyvtárral egy szinten lévő fájlokra rá lát a rendszer, szóval amennyiben a fájlokat nem útvonallal adjuk meg, úgy alapértelmezetten itt keresi az output és input fájlokat.

A program nem csak PVM-konzolból, hanem futatható állományként is indítható, ha belenavigálunk a $~/pvm3/bin/LINUX64/$ könytárba és az ott elkészített szimbolikus linkeket szólítjuk meg, az előbb említettekkel analóg módon:

\verb|$ ~/pvm3/bin/LINUX64/master <szám> <input> <output>|

Figyeljünk a paraméterek helyességére illetve teljességére továbbá a bemeneti adat formátumának helyességére, mert a program külön ellenőrzést erre vonatkozóan nem végez. Ha nincs megfelelő számú paraméter a program kilép 1-es hibakóddal.

\section{Fejlesztői dokumentáció}

A program két C++ 11. szabványnak megfelelő fordítási egységből áll. A főprogram, a főszál a $master.cpp$ állományban, míg a megoldás magja a gyerekfolyamatokat implementáló $child.cpp$ állományban található amely a rekurziót fogja adni a megoldáshoz.

\subsection{Megoldási mód}


A főszál feladata az adatok beolvasása, és a gyermek-lánc elindítása, végül az a válasz fogadása, majd a válasz megfelelő fájlba történő kiírása.

A gyemek feladata, hogy fogadja a megfelelő üzeneteket, és adatoktól függően vissza üzen a szülőnek az alapesetben foglalt válaszokkal, vagy indít két újabb gyeremeket, a kontextusnak megfelelő módosított paraméterekkel, bevárja a válaszukat és úgy üzen vissza a szülőjének.

\subsection{Implementáció}
	\subsubsection{Master}
    
    A fentebb leírt feladatot 4 alfeladatra bonthatjuk:
    \begin{enumerate}
    \item Bemeneti paraméterek egyszerű validálása és inicializálása megfelelő változókba
    \item Adatok beolvasása
    \item Gyermek folyamat indítása, üzenet küldése, illetve fogadása
    \item Eredmény kiírása
	\end{enumerate}
    
    Ugyanakkor implementálás során, optimalizálható a kód egyes részegységek összefogásával.
    \begin{itemize}
    \item A gyerekfolyamatot már a beolvasás előtt létrehozzuk, s így miután beolvassuk az adatokat rögtön el is küldjük.
    \item Ugyanígy a kiírás esetén is, egyből a válasz fogadásakor írjuk az eredmény fájlt, elkerülve ezzel felesleges másolásokat.
    \end{itemize}
    
    A folyamatot tehát, azzal kezdjük, hogy megvizsgáljuk, hogy minden paraméter adott-e az indításhoz, ha nem akkor rögtön kilépünk 1-es kóddal. (\verb|exit(1)|) 
    Ezután létrehozzuk a leendő egy darab kezdő gyerek-folyamatot, illetve vizsgáljuk annak valid létrejöttét, annak érdekében, hogy amennyiben nincs mód új gyermek létrehozására, úgy a program se használjon felesleges változókat. Ilyen hiba esetén a PVM-et is lezárjuk (\verb|pvm_exit()|), illetve 2-es hibakóddal kilépünk a programból. (\verb|exit(2)|)
    Amennyiben minden paraméter adott és a gyermek is létrejött inicializáljuk  a paramétereket megfelelő változókba. A $k$ egy egész értékű változó, míg a $input$ és $output$ \verb|std::string| típusúak.
    Ezután elküdjük a gyermeknek $k$-t, majd beolvasásra kerül $N$ (egy pozitív egész szám (\verb|unsigned int|-ként implementálva, hiszen ez jelképezi az átadandó halmaz elemszámát)), amit szintén elküldünk, hogy a gyermek tudja, hogy hány adatot kell fogadnia a következő üzenetben. Inicializáljuk a halmazunkat reprezentáló $N$ elemű vektort (\verb|std::vector<int>|), és feltöltjük a beolvasott értékekkel, majd elküldjük ezt is a gyereknek.
    Ezután már csak a választ kell megvárni, amelyet egy karakter formájában fogadunk (ez a legkisebb méretű üzenettípus, ami rendelkezésre áll, hogy egy boolean típust szimuláljunk), amitől függően a megfelelő szöveget írjuk az $output$ fájlba.

	\subsubsection{Child}
    
    A gyermekfolyamatok feladatát maga a \textit{Divide \& Conquer} módszer adja: feldaraboljuk a problémát egyszerűbb alproblémákra, amiket addig bontogatunk szét rekurzívan, ameddig triviális alapesetek segítségével megoldhatóvá válik a probléma.
    Így a megoldási elv a következő:
    
    Fogadunk egy üzenetet a szülőfolyamattól, amely egy egész szám lesz $k$. Amennyiben ez a szám 0, úgy alapeset szerint a válasz hamis, ha nem akkor fogadunk egy újabb üzenetet, ami egy pozitív egész szám, $n$ lesz. Ha $n$ 0, azaz üres halmazt foganánk a következő üzenetben, úgy már most tudunk válaszolni, hogy üres halmaz esetén (és mert $k$ nem 0) a válasz hamis, egyés esetben pedig a rekurziós ágba lépünk.
    
    A rekurziós ág elején fogadjuk az $n$ darab egész számot a szülőtől, amiket egy $n$ elemű vektorban tárolunk, majd megkísérelünk létrehozni 2 gyerek folyamatot. Amennyiben létrejöttek, az első gyereknek elküldjük a $k$-t, $n-1$-et, továbbá azt az $n-1$ elemszámú halmazt reprezentáló vektort, amiből kivesszük "véletlenszerűen" az utolsó elemet.\footnote{Mivel halmaz elemei között nincs sorrend, ezért a specifikáció véletlenszerűséget fogalmaz meg kiválasztásnál, de az algoritmus implementálása szempontjából hatékonyabb a vektor utolsó elemét, tekinteni mindig a "véletlen" kiválasztott elemnek}
A második gyereknek pedig elküldjük a $k-tail$ egész értéket, ahol $tail$ a vektorunk utolsó egész eleme, $n-1$-et, továbbá a vektor utolsó eleme nélküli vektort (Ez a \verb|pop_back| metódussal került megvalósításra, mivel mindkét esetben a pop-olt vektorra van szükség).

Ezután fogadjuk a választ mindkét gyerektől, s válaszolunk a szülőfolyamatnak. (Amennyiben az első gyerek igazat ad, a másodiktól üzenetet nem is szükséges fogadni, hiszen azonnal tudunk üzenni a szülőfolyamatnak, hogy a válasz igaz, egyéb esetben meg kell várni a másik gyerek válasza lesz a szülő válasza. (Ha igaz, akkor igaz, ha hamis akkor hamis))
    
\subsection{Fordítás menete}

A programunk forráskódját a \verb|master.cpp| és a \verb|child.cpp| állomány tartalmazza. A fordításhoz szükséges \verb|Makefile.aimk| állomány amely a fordítási információkat tartalmazza a forrásállományokkal egy szinten található.
A fordításhoz a \textit{g++} fordítóprogram szükséges, illetve az \textit{aimk} program, aminek segítségével könnyen kialakítható a PVM fordításhoz szükséges környezet. A $Makefile$-ban meghatározott információk alapján lefordul az összes benne meghatározott fordítási komponens, amennyiben a $~/pvm3/src/$ könyvtárban kiadjuk a következő parancsot:

\verb|aimk| \\
Ezután, ha futtatható állományokat szeretnénk kapni, létre kell hozni a szimbolikus linkeket, ugyanerről a hyleről kiadva a következő parancsot:

\verb|aimk links|\\
Ha mindkét parancs hibamentesen lefutott, lefordult a kód és a megfelelő linkek is létrejöttek. Így a konzolból (vagy intézőből) futtatható a program.

\subsection{Tesztelés}

A program tesztelése során különböző méretű bemeneti fájlokkal, illetve paraméterekkel futtattam. (Az \verb|atlasz.elte.hu| szerver 20 szálas korlátjából adódó lehetőségekhez mérten)
A programom minden esetben a tőle elvárt kimenetet állította elő, így a tesztesetek alapján helyesnek gondolhatjuk a működését.

\begin{enumerate}
    \item Hiányzó paraméterek tesztesete
    \item k = 0 triviálisan igaz eset
    \item |n| = 0 triviálisan hamis eset
    \item k összegű részhalmaz létezése, n=1, n=2, n=3, n=4 esetekre \footnote{ \label{note1} n=5 eset már 2$^n$ > 20 szálat produkálna}    \item k összegű részhalmaz nem létezése, n=1, n=2, n=3, n=4 esetekre $^{\ref{note1}}$
\end{enumerate}

A tesztelést az \verb|atlasz.elte.hu| szerveren végeztem. Itt az $atlasz$ fejgép szimbolizálja az 1 processzormagot.
A tesztelés során sorban adogattam hozzá a blade-ket, egészen 7 darab hostig. (8 hostot már nem mindig engedett a szerver)
Az itt mért futásidők egy összefoglaló táblázatban:

\begin{itemize}
	\item $k :$ A számláló érték
    \item $n :$ Az aktuális halmaz elemszáma ($|n|$)
    \item $T_1 : k = 0$ eset, triviális igaz válasz, 1 gyermek
    \item $T_2 : n = 0$ eset triviálisan hamis eset, 1 gyermek
    \item $T_{3_i} : n = i$ illetve a $k$ egy olyan érték amelyre a program igaz választ ad.($i=1..4$) 
    \item $T_{4_i} : n = i$ illetve a $k$ egy olyan érték amelyre a program hamis választ ad.($i=1..4$)
    \item $N/A :$ Azt jelenti, hogy a szerver nem volt képes kezelni, ezt az esetet.
\end{itemize}

\begin{adjustbox}{max width=\textwidth}
	\begin{tabular}{| l | r | r | r | r | r | r | r | r | r | r |}
      \hline
       & $T_1$ & $T_2$ & $T_{3_1}$ & $T_{4_1}$ & $T_{3_2}$ & $T_{4_2}$ & $T_{3_3}$ & $T_{4_3}$ & $T_{3_4}$ & $T_{4_4}$ \\ \hline
      1 mag  & $\sim$0.1121 mp & $\sim$0.1118 mp & $\sim$0.0229 mp & $\sim$0.0272 mp & $\sim$0.0345 mp & $\sim$0.0345 mp & N/A & N/A & N/A & N/A  \\ \hline
      2 mag & $\sim$0.0116 mp & $\sim$0.0025 mp & $\sim$0.0148 mp & $\sim$0.0151 mp & $\sim$0.0177 mp & $\sim$0.0263 mp & $\sim$0.0384 mp & $\sim$0.038 mp & N/A & N/A \\ \hline
      3 mag & $\sim$0.00358 mp & $\sim$0.0114 mp & $\sim$0.015 mp & $\sim$0.0142 mp & $\sim$0.0178 mp & $\sim$0.0262 mp & $\sim$0.0293 mp & $\sim$0.0375 mp & $\sim$0.0437 mp & $\sim$0.042 mp \\ \hline
      7 mag & $\sim$0.00357 mp & $\sim$0.00386 mp & $\sim$0.00816 mp & $\sim$0.0152 mp & $\sim$0.0101 mp & $\sim$0.0189 mp & $\sim$0.0297 mp & $\sim$0.0313 mp & $\sim$0.033 mp & $\sim$0.0343 mp \\
      \hline
	\end{tabular}
\end{adjustbox}
\\
\\
Látható, hogy az egyes teszteset típusoknál mekkora gyorsulás érhető el a számítási egységek számának növelésével, és szinte minden esetben közel 3-szor gyorsabb volt a 7 magos megoldás mint a szekvenciális számítással ekvivalensnek tekinthető 1 magos verzió.

\end{document}