\documentclass[10pt]{article}
\usepackage{makeidx}
\usepackage{multirow}
\usepackage{multicol}
\usepackage[dvipsnames,svgnames,table]{xcolor}
\usepackage{epstopdf}
\usepackage{ulem}
\usepackage{hyperref}
\usepackage{amsmath}
\usepackage{amssymb}
\author{Mikus M\'{a}rk Istv\'{a}n}
\title{}
\usepackage[paperwidth=595pt,paperheight=841pt,top=70pt,right=70pt,bottom=70pt,left=70pt]{geometry}

\makeatletter
	\newenvironment{indentation}[3]%
	{\par\setlength{\parindent}{#3}
	\setlength{\leftmargin}{#1}       \setlength{\rightmargin}{#2}%
	\advance\linewidth -\leftmargin       \advance\linewidth -\rightmargin%
	\advance\@totalleftmargin\leftmargin  \@setpar{{\@@par}}%
	\parshape 1\@totalleftmargin \linewidth\ignorespaces}{\par}%
\makeatother 

% new LaTeX commands


\begin{document}

\begin{center}
	{\Huge Orsi beadand\'{o} feladat  - dokument\'{a}ci\'{o}}
\end{center}

\begin{center}
	{\large 2016.11.05.}
\end{center}

{\large Valamikor, a nem t\'{u}l t\'{a}voli j\"{o}v\H{o}ben az Informatikai Kar
t\'{u}ln\H{o}tt a programoz\'{o}k\'{e}pz\'{e}s keretein. A hatalmas
t\'{u}ljelentkez\'{e}s \'{e}s munkaer\H{o}piaci ig\'{e}ny miatt
sz\"{u}ks\'{e}gess\'{e} v\'{a}lt tov\'{a}bbi hallgat\'{o}k felv\'{e}tele, akik
k\'{e}s\H{o}bb a megszerzett tud\'{a}st versenyk\"{o}r\"{u}lm\'{e}nyek
k\"{o}z\"{o}tt tudj\'{a}k hasznos\'{\i}tani. Vil\'{a}goss\'{a} v\'{a}lt, hogy a
profi fejleszt\H{o}k \'{e}s a hivat\'{a}sos felhaszn\'{a}l\'{o}k
egy\"{u}ttm\H{u}k\"{o}d\'{e}se elengedhetetlen. B\'{a}r tov\'{a}bbra is hatalmas
\'{e}rdekl\H{o}d\'{e}s \"{o}vezte a programoz\'{o}i k\'{e}pz\'{e}st, \'{a}m egyre
t\"{o}bben j\"{o}ttek gamer amb\'{\i}ci\'{o}kkal. Az egyetem \'{u}gy
d\"{o}nt\"{o}tt, hogy \'{u}j Kart ind\'{\i}t a j\'{a}t\'{e}kos
k\"{o}z\"{o}ss\'{e}g ig\'{e}nyeinek kiszolg\'{a}l\'{a}s\'{a}ra, hogy a profi
eSportol\'{o}k k\"{o}z\"{o}ss\'{e}ge min\H{o}s\'{e}gi oktat\'{a}st kaphasson az
ELTE-n bel\"{u}l.}

{\large A k\"{u}l\"{o}nb\"{o}z\H{o} tansz\'{e}kekhez, az oktat\'{o}i \'{e}s
hallgat\'{o}i l\'{e}tsz\'{a}mhoz, valamint a g\'{e}ptermek
felszerelts\'{e}g\'{e}hez azonban sz\"{u}ks\'{e}ges inform\'{a}ci\'{o} a
megfelel\H{o} j\'{a}t\'{e}kost\'{a}borral ar\'{a}nyos er\H{o}forr\'{a}sok
biztos\'{\i}t\'{a}sa. A proginf.-es k\'{e}pz\'{e}sben r\'{e}sztvev\H{o}, B
szakir\'{a}nyos di\'{a}kokra b\'{\i}zz\'{a}k a kisz\'{a}m\'{\i}t\'{a}s\'{a}t
annak, hogy milyen s\'{u}ly\'{u} a k\"{u}l\"{o}nb\"{o}z\H{o} j\'{a}t\'{e}kok
ir\'{a}nti \'{e}rdekl\H{o}d\'{e}s, hogy ehhez m\'{e}rve igaz\'{\i}thass\'{a}k a
tant\'{a}rgyi feloszt\'{a}st.}

{\large A felv\'{e}teliz\H{o}k k\"{o}z\"{o}tt ez\'{e}rt el\H{o}zetes
felm\'{e}r\'{e}st v\'{e}geztek, ki milyen j\'{a}t\'{e}kokkal j\'{a}tszik
szabadidej\'{e}ben. Az \'{\i}gy kapott adathalmaz k\"{o}zkedvelt j\'{a}t\'{e}kai
k\"{o}z\"{u}l~\textit{M}~kapott d\"{o}nt\H{o} mennyis\'{e}g\H{u} szavazatot,
\'{\i}gy a m\'{a}sodik k\"{o}rben ezekre sz\H{u}k\"{u}lt a k\'{e}rd\H{o}\'{\i}v,
melyben minden hallgat\'{o} v\'{a}laszolt, hogy egy h\'{e}ten mennyi id\H{o}t
(percet) t\"{o}lt az egyes j\'{a}t\'{e}kokkal. Term\'{e}szetesen voltak, akik nem
csup\'{a}n egyetlen m\H{u}fajnak h\'{o}doltak, \'{\i}gy el\H{o}fordult, hogy
t\"{o}bb kit\"{o}lt\'{e}s is \'{e}rkezett ugyan att\'{o}l a hallgat\'{o}t\'{o}l.
A n\'{e}vtelens\'{e}g, de egyszeri v\'{a}laszad\'{a}s fenntart\'{a}s\'{a}nak
\'{e}rdek\'{e}ben a kit\"{o}lt\H{o}ket NEPTUN-k\'{o}ddal azonos\'{\i}tott\'{a}k.
A j\'{a}t\'{e}kokat egy-egy sz\'{o}k\"{o}zt nem tartalmaz\'{o}, sz\"{o}veg
t\'{\i}pus\'{u} adat reprezent\'{a}lja (pl. "WoW" vagy "LoL").}

	
\section{Feladat}
	

A bemenet --~input.txt~-- els\H{o} sor\'{a}ban tartalmazza a sz\'{o}k\"{o}zzel
elv\'{a}lasztott~N~\'{e}s~M~pozit\'{\i}v eg\'{e}szeket -- ahol
k\"{o}vetkez\H{o}~N~sor\'{a}ban pedig egy-egy neptunk\'{o}d, j\'{a}t\'{e}kn\'{e}v
\'{e}s percben elt\"{o}lt\"{o}tt id\H{o} (eg\'{e}sz sz\'{a}m) sz\'{o}k\"{o}zzel
elv\'{a}lasztott h\'{a}rmas\'{a}t

\begin{center}
	{\raggedright
	{\small N M}
	}
	
	{\raggedright
	{\small NEPTUN\_1 gameazon perc   - az 1. v\'{a}lasz adatai}
	}
	
	{\raggedright
	{\small NEPTUN\_2 gameazon perc   - az 2. v\'{a}lasz adatai}
	}
	
	{\raggedright
	{\small ...}
	}
	
	{\raggedright
	{\small ...}
	}
	
	{\raggedright
	{\small ...}
	}
	
	{\raggedright
	{\small NEPTUN\_N gameazon perc   - az N. v\'{a}lasz adatai}
	}
\end{center}

{\normalsize A f\H{o}folyamat olvassa be az adatokat,
ind\'{\i}tson~\textit{M}~gyerekfolyamatot, majd minden gyerekfolyamathoz
t\'{a}rs\'{\i}tson egy-egy j\'{a}t\'{e}khoz tartoz\'{o} adathalmazt. (A
j\'{a}t\'{e}kok sz\'{a}ma pontosan~\textit{M}, \'{\i}gy minden j\'{a}t\'{e}k
adat\'{a}t p\'{a}rhuzamosan kell kisz\'{a}molni.) A gyerekfolyamatok dolga
meg\'{a}llap\'{\i}tani, hogy az adott j\'{a}t\'{e}kkal mennyi id\H{o}t
t\"{o}lt\"{o}ttek \"{o}sszesen (percben) a v\'{a}laszad\'{o}k, valamint az
\'{a}tlagos j\'{a}t\'{e}kid\H{o} kisz\'{a}m\'{\i}t\'{a}sa az adott
j\'{a}t\'{e}khoz (eg\'{e}sz percre lefel\'{e} kerek\'{\i}tve - floor(..) ). Ezt a
k\'{e}t adatot k\"{u}ldje vissza a sz\"{u}l\H{o}folyamatnak.}

{\normalsize A f\H{o}folyamat ezek ut\'{a}n a j\'{a}t\'{e}kokhoz tartoz\'{o},
\"{o}sszesen \'{e}s \'{a}tlagosan elt\"{o}lt\"{o}tt id\H{o}t sz\'{o}k\"{o}zzel
elv\'{a}lasztva a J\'{A}T\'{E}KOK azonos\'{\i}t\'{o}ja ALAPJ\'{A}N
BET\H{U}RENDBEN \'{\i}rja az~\textit{output.txt}~kimeneti f\'{a}jlba.}

\begin{center}
	{\raggedright
	{\small //egy p\'{e}lda sor az output f\'{a}jlb\'{o}l:}
	}
	
	{\raggedright
	{\small //WoW 6000 573}
	}
\end{center}

{\raggedright
	{\normalsize Tov\'{a}bbi elv\'{a}r\'{a}sok:}
}

{\normalsize A megold\'{a}saitokat egyetlen ZIP f\'{a}jlba
t\"{o}m\"{o}r\'{\i}tve t\"{o}lts\'{e}tek fel! K\'{e}r\"{u}nk benneteket, hogy
csak a sz\"{u}ks\'{e}ges forr\'{a}sf\'{a}jl(oka)t rakj\'{a}tok bele, ne teljes
projektet (.exe semmik\'{e}pp sem)!}

{\normalsize A k\"{u}l\"{o}nb\"{o}z\H{o} technol\'{o}gi\'{a}k miatt
szeretn\'{e}nk mindenkit megk\'{e}rni, hogy az al\'{a}bbiak vegye figyelembe a
felt\"{o}lt\'{e}s sor\'{a}n:}

{\raggedright
{\normalsize PVM: A k\'{e}t .cpp f\'{a}jlt, \'{e}s a Makefile.aimk-t
egyar\'{a}nt csomagolj\'{a}tok be, a sz\"{u}l\H{o}folyamathoz tartoz\'{o} k\'{o}d
'master.cpp' -nek legyen elnevezve.}
}

{\raggedright
{\normalsize (C++11: Nincs k\"{u}l\"{o}n megk\"{o}t\'{e}s, tetsz\H{o}leges
f\'{a}jln\'{e}vvel rendelkezhet.)}
}

\section{Felhaszn\'{a}l\'{o}i dokument\'{a}ci\'{o}}
\subsection{K\"{o}rnyezet}


A program t\"{o}bb platformon futtathat\'{o}, nincsen dinamikus
f\"{u}gg\H{o}s\'{e}ge. Telep\'{\i}t\'{e}sre nincs sz\"{u}ks\'{e}g, elegend\H{o} a
futtathat\'{o} \'{a}llom\'{a}ny elhelyezni a sz\'{a}m\'{\i}t\'{o}g\'{e}pen.

\subsection{Haszn\'{a}lat}

A program elind\'{\i}t\'{a}sa egyszer\H{u}, mivel nem v\'{a}r parancssori
param\'{e}tereket, \'{\i}gy parancssoron k\'{\i}v\"{u}l is lehet futtatni. A
f\'{a}jl mellett kell elhelyezni az input.txt-t 1 f\'{a}jl, melyet feldolgoz
\'{e}s az eredm\'{e}nyt az ''output.txt'' nev\H{u} f\'{a}jlba \'{\i}rja, a
bemeneti sorrend alapj\'{a}n. Egy lehets\'{e}ges bemenetet tartalmaz a
mell\'{e}kelt input.txt tesztf\'{a}jl. Saj\'{a}t bemeneti f\'{a}jlok eset\'{e}n
fontos, hogy a feladatban megadott szempontok alapj\'{a}n \'{\i}rjuk az adatokat
az inputf\'{a}jlba, mivel a program kil\'{e}p ha az adatok a f\'{a}jlban nem
helyesek.


\section{Fejleszt\H{o}i dokument\'{a}ci\'{o}}
\subsection{A megold\'{a}s m\'{o}dja}

{\normalsize
{\raggedright
A k\'{o}dot logikailag k\'{e}t r\'{e}szre bonthatjuk, egy f\H{o} \'{e}s t\"{o}bb
gyermekfolyamatra. A f\H{o}folyamatot a master.cpp-ben szerepl\H{o} main
f\"{u}ggv\'{e}ny fogja megval\'{o}s\'{\i}tani. Feladata, hogy beolvassa az
inputf\'{a}jl tartalm\'{a}t, majd a feladatban meghat\'{a}rozott m\'{o}don
l\'{e}trehozzon alfolyamatokat.
}
}

{\normalsize
Az N sor szavazat, illetve a M fajta csoport, meghat\'{a}roz\'{o} k\'{e}t
pozit\'{\i}v eg\'{e}sz sz\'{a}m, beolvas\'{a}sa ut\'{a}n. Egy olyan
strukt\'{u}r\'{a}ba olvassuk be az adatokat, amely t\'{a}mogatja, az adatok
csoportos\'{\i}t\'{a}s\'{a}t kulcs szerint. Erre sz\"{u}ks\'{e}g lesz, amikor
kiosztjuk a gyerekfolyamatoknak, a r\'{e}szfeladatokat. Egy gyermek
r\'{e}szfeladata, hogy meghat\'{a}rozza a hozz\'{a} tartoz\'{o} kateg\'{o}ria,
\"{o}sszes illetve \'{a}tlagos idej\'{e}t, majd vissza\"{u}zenje a
sz\"{u}l\H{o}nek.

A f\H{o}folyamat az alfolyamatokb\'{o}l visszanyert adatot, rendezett
form\'{a}ban az output.txt f\'{a}jlba \'{\i}rja bele.
}

\subsection{Implement\'{a}ci\'{o}}

{\normalsize
{\raggedright
A C++ megval\'{o}s\'{\i}t\'{a}s sor\'{a}n, a nyelvi elemek k\"{o}z\"{u}l, a
legmegfelel\H{o}bb adatstrukt\'{u}ra az \textbf{std::map, } azon bel\"{u}l is a
\textbf{std::multimap}, mivel a map egy asszociat\'{\i}v adatstrukt\'{u}ra, amely
tartja a rendezetts\'{e}g\'{e}t a kulcsai szerint n\"{o}vekv\H{o} sorrendben.
(J\'{a}t\'{e}kn\'{e}vsorrend) Tov\'{a}bb\'{a} egyedi kulcsokkal kell ell\'{a}tni
az egyes elemeket. Ezt kik\"{u}sz\"{o}b\"{o}lend\H{o}en a multimap m\'{a}r
megenged t\"{o}bb elemet ugyanahhoz a kulcshoz, \'{e}s ugyan\'{u}gy tartja a
rendezetts\'{e}get.
}}

{\normalsize
{\raggedright
Az implement\'{a}ci\'{o} a master.cpp sz\"{u}l\H{o}folyamatb\'{o}l \'{e}s a
child.cpp gyerekfolyamatok main f\"{u}ggv\'{e}nyeinek t\"{o}rzs\'{e}ben
tal\'{a}lhat\'{o}.
}}

\subsection{Ford\'{\i}t\'{a}s}

{\normalsize
A program forr\'{a}sk\'{o}dja a master.cpp illetve a child.cpp. A program
ford\'{\i}t\'{a}s\'{a}hoz k\"{o}vetelm\'{e}ny egy c++11 szabv\'{a}nyt
t\'{a}mogat\'{o} ford\'{\i}t\'{o}program megl\'{e}te a rendszeren.

A fejleszt\'{e}s, ill. tesztel\'{e}s sor\'{a}n a g++ ford\'{\i}t\'{o}t
haszn\'{a}ltam. A ford\'{\i}t\'{o}ban speci\'{a}lis
be\'{a}ll\'{\i}t\'{a}sk\'{e}nt a stringeket t\'{a}mogat\'{o} -Wno-write-strings
kapcsol\'{o}t is alkalmazni kell.

Az --std=c++11 kapcsol\'{o}m is sz\"{u}ks\'{e}ges, mert alap\'{e}rtelmezetten
r\'{e}gebbi c++ szabv\'{a}nyt t\'{a}mogat a ford\'{\i}t\'{o}.

A programhoz csatolt Makefile.aimk f\'{a}jl seg\'{\i}ts\'{e}get ny\'{u}jt a
ford\'{\i}t\'{a}shoz, mert mag\'{a}ba foglalja a ford\'{\i}t\'{a}shoz
sz\"{u}ks\'{e}ges inform\'{a}ci\'{o}kat.
}
\newline
\newline
\subsection{Tesztel\'{e}s}

{\normalsize
{\raggedright
A program helyes m\H{u}k\"{o}d\'{e}s\'{e}t, k\"{u}l\"{o}nb\"{o}z\H{o} a
feladathoz kapott tesztel\'{e}si anyagok seg\'{\i}ts\'{e}g\'{e}vel v\'{e}geztem.
A mint\'{a}k \'{a}ltal\'{a}nos, \'{e}s p\'{a}r sz\'{e}ls\H{o}s\'{e}ges eseteket
dolgoznak fel.
}
}
\newline
\newline

\begin{enumerate}
	\item Input\_1.txt
	\begin{center}
		{\raggedright
			11 7
		}
		
		{\raggedright
			D482J4 LoL 606
		}
		
		{\raggedright
			D482J4 OW 857
		}
		
		{\raggedright
			BZJW9I HotS 2551
		}
		
		{\raggedright
			LFQSJN DotA 2137
		}
		
		{\raggedright
			LFQSJN LoL 1667
		}
		
		{\raggedright
			U4NOA1 CoD 1354
		}
		
		{\raggedright
			U4NOA1 MC 1277
		}
		
		{\raggedright
			TYGS6Y CoD 1442
		}
		
		{\raggedright
			AEJRLZ GTAV 182
		}
		
		{\raggedright
			ED06UQ DotA 1264
		}
		
		{\raggedright
			6VSJ3K MC 193
		}
	\end{center}

	\item Output\_1.txt
	\begin{center}
{\raggedright
CoD 2796 1398
}

{\raggedright
DotA 3401 1700
}

{\raggedright
GTAV 182 182
}

{\raggedright
HotS 2551 2551
}

{\raggedright
LoL 2273 1136
}

{\raggedright
MC 1470 735
}

{\raggedright
OW 857~857
}
	\end{center}

	\item Input\_2.txt
\begin{center}

{\raggedright
6 5
}

{\raggedright
9MQGI1 MC 61
}

{\raggedright
9MQGI1 WoW 1564
}

{\raggedright
NC2XBQ LoL 2378
}

{\raggedright
NC2XBQ GTAV 733
}

{\raggedright
EMV17T WoW 387
}

{\raggedright
JXJJTU HotS 352
}
	\end{center}

	\item Output\_2.txt
	\begin{center}
{\raggedright
GTAV 733 733
}

{\raggedright
HotS 352 352
}

{\raggedright
LoL 2378 2378
}

{\raggedright
MC 61 61
}

{\raggedright
WoW 1951 975
}

	\end{center}

	\item Input\_3.txt
\begin{center}

{\raggedright
2 2
}

{\raggedright
PXXMYG OW 1319
}

{\raggedright
PXXMYG CoD 2440
}
	\end{center}

	\item Output\_3.txt
\begin{center}

{\raggedright
CoD 2440 2440
}


{\raggedright
OW 1319 1319
}
	\end{center}


	\item Input\_4.txt\hspace{15pt}
\begin{center}

{\raggedright
1 1
}

{\raggedright
PXXMYG WoW 1319
}
	\end{center}

	\item Output\_4.txt
\begin{center}

{\raggedright
WoW 1319 1319
}

	\end{center}
\end{enumerate}

\end{document}