% BEGIN -- SETUP DOCUMENT (OVERLEAF) --
\documentclass[a4paper,12pt]{amsart}%
\usepackage{amsmath}
\usepackage{amsfonts}
\usepackage{amssymb}
\usepackage{nopageno}
\usepackage[T1]{fontenc}
\usepackage[utf8]{inputenc}
\usepackage{enumerate}
\newtheorem{theorem}{Theorem}
\newtheorem*{tetel}{T\'etel}
\usepackage[british]{babel}
\usepackage{csquotes}
\usepackage[backend=biber,style=apa]{biblatex}
\DeclareLanguageMapping{british}{british-apa}
\usepackage[pdftex]{graphicx}
\addbibresource{ref.bib}
\let\cite\parencite
\begin{document}
% END -- SETUP DOCUMENT (OVERLEAF) --

% BEGIN -- SETUP DOCUMENT (TEXMAKER) --
% \documentclass[a4paper,12pt]{article}
% \usepackage{hyperref}
% \usepackage{apacite}
% \begin{document}
% \bibliographystyle{apacite}
% END -- SETUP DOCUMENT (TEXMAKER) --

\title{ ORSI. 3. Beadandó feladat }
\author{Mikus Márk István}
\date{Nov 27, 2016}
\maketitle

\section{Feladat}
A bemenet input.txt első sorában egy N pozitív egész olvasható, ennyi diáknak kell biztosítani termet, míg a következő N sorban a hallgatók NEPTUN-kódjai, azaz N string követi egymást, ez mutatja a jelentkezési sorrendet.\\
\\
Egy lehetséges bemeneti fájl:\\
7\\
OSAVH1\\
T0NDJB\\
4S1UPL\\
AXKAW4\\
22TQP7\\
NM8VPS\\
PJVNEU\\
\\
Egy lehetséges kimeneti fájl:
\\
22TQP7\\
4S1UPL\\
AXKAW4\\
NM8VPS\\
OSAVH1\\
PJVNEU\\
T0NDJB\\

\subsection{Megjegyzések.}


\section{Felhasználói dokumentáció}

\subsection{Környezet}

A program platformfüggetlen, a program futtatásához elég a futtatható állományok elindítása. 
Telepítésre nincs szükség. 

\subsection{Használat}

A futtatáskor parancssori paramétert nem vár a program.

A futtatható állomány mellé kell helyezni az input.txt fájlt.Illetve ugyanide kell létrehozni egy üres output.txt nevű fájlt.
A program az input fájlból fog dolgozni, majd az output fájlba írja az eredményt, ha tudja.
Ha nem tudja manipulálni a megfelelő fájlokat, a program hibajelzést ad.

\section{Fejlesztői dokumentáció}
\subsection{A megoldás módja}
A kódot logikailag két részre szeparáltuk, mester és gyermek folyamatra. A master.cpp forrásfájlban található főfüggvény (main) reprezentálja a mesterszálat, a programkonstrukciónkban. A gyermekfolyamatok implementálása a child.cpp-ben kap helyet.
Itt kerül implementálásra, a megoldás magját adó rekurzív összefédülő renedezés-algoritmus. Illetve az ahhoz tartozó két függvény: a merge, amellyel sorozatokat rendezünk, és a mergeSort, amely az oszd meg és uralkodj elvén, szétosztja két újabb gyermek között az input sorozat felét.

A mester szálon lévő program kizárólagos feladata, az input fájlból az adatok beolvasása, majd azok továbbítása a gyerek felé.
A végeredmény megkapásakor pedig, azt egy output fájlba kell írnia.

\subsection{Implementáció}
Az implementálás során a fentebb említett szemlélet volt a mérvadó.
A program hibakezelést végez, a a fájlokkal való kommunikáció során(, majd később a pvm spwan metódusa kapcsán is).
Az inputot a program egy string tömbbe pakolja be.Amely n elemű.
A string tömböt egy szintén n-es forciklussal tölti fel.

Majd ugyanezen végigiterálva elküldjük a gyereknek a méretét, majd magát a tömböt is.
A főfolyamat ezek után már csak visszavár egy n méretű string tömböt, amit miután sikeresen fogad, beleír az output fájlba.

A gyermekfolyamat törzse lényegében az üzenetek fogadásából áll, majd az eredmény előőállításából és végül az eredmény visszaküldéséből.

Az eredmény előállítását a mergeSort nevű 3 paraméteres függvény végzi, amely rekuzívan hívható, s minden híváskor létrehoz 2 gyermeket, akiknek, a megfelelő szeletek méretétt és magát a szeletet is átküldi, majd viszontvárja az eredményt.

\subsection{Fordítás}
A program forráskódja a már említett master.cpp illetve a child.cpp. A program fordításához követelmény egy c++11
szabványt támogato fordítóprogram megléte a rendszeren.
A fejlesztés, ill. tesztelés során a g++ fordítót használtam. A fordítóban speciális beallításként a
stringeket támogató -Wno-write-strings kapcsolót is alkalmazni kell.
Az std=c++11 kapcsoló is szükséges, mert alapertelmezetten régebbi c++ szabvanyt támogat a fordító.
A programhoz csatolt Makefile.aimk fajl segítséget nyújt a fordítashoz, mert magába foglalja a fordításhoz szükseges összes információt.
Ennek segítségévek elég meghívni az aimk, parancsot a forrássájlok könyvtárában állva. Majd miután végezett, futtatható is a program.

(PVM futtatása esetén: mostmár kiadható a pvm parancs, majd a pdm démonból indítható spawn-nal a master folyamat.)

\subsection{Tesztelés}	

A tesztelést és a futtatást, PVM környezetben végeztem.
Ehhez hozzáférheetőséget kaptam az ELTE Atlasz nnevű kiszolgálójához.
A merge metódust, amely az összefésülést végzi, a kari honlapokon talált segédanyagok alapján implementáltam.
Ahogy a mergeSort metódust a feladathoz kapcsolt, rekurzív leírás alapján került megvalósításra.

A merge metódushoz gyakorlati számnításokat végeztem a szélsősélges tesztesetekre. (1,2 eleműek összefűzése)
A mergeSort metódust, pedig futtatások során teszteltem, figyelve arra, hogy felépülésre kerül-e a megfelelő bejárási fa.

Megjegyzés:\\
Sajnálatos módon az atlasz kiszámíthatatlan viselkedést produkált a fejlesztés alatt.

% BEGIN -- END DOCUMENT (OVERLEAF) --
\printbibliography
\end{document}
% END -- END DOCUMENT (OVERLEAF) --

% BEGIN -- END DOCUMENT (TEXMAKER) --
% \bibliography{ref}{}
% \end{document}
% END -- END DOCUMENT (TEXMAKER) --